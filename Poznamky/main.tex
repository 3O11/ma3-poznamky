\documentclass[11pt]{article}

\usepackage[czech]{babel}
\usepackage{a4wide}
\usepackage[utf8]{inputenc}
\usepackage[T1]{fontenc}
\usepackage{fancyhdr}
\usepackage{amssymb}
\usepackage{amsthm}
\usepackage{amsmath}
\usepackage{mathtools}
\usepackage{mleftright}
\usepackage{subfig}

% forcing footnotes to be at the very bottom
\usepackage[bottom]{footmisc}

\usepackage{hyperref}
\usepackage{titlesec}
%This has to be the last
\usepackage{subfiles}

\usepackage{geometry}
\geometry{
    a4paper,
    total={170mm,257mm},
    right=20mm,
    left=20mm,
    top=30mm,
    bottom=20mm,
}

\DeclareMathOperator{\rank}{rank}
\DeclareMathOperator{\Span}{span}

\newtheoremstyle{nontheoremstyle}{1em}{1em}{}{}{\bfseries}{:}{.5em}{}
\newtheoremstyle{theoremstyle}{1em}{1em}{\it}{}{\bfseries}{:}{.5em}{}

\theoremstyle{nontheoremstyle}
\newtheorem*{definition}{Definice}
\newtheorem*{example}{Příklad}
\renewenvironment{proof}{{\noindent\bfseries Důkaz:}}{\qed}
\newtheorem*{intuition}{Intuice}
\newtheorem*{remark}{Poznámka}
\newtheorem*{consequence}{Důsledek}
\newtheorem*{observation}{Pozorování}

\DeclarePairedDelimiter\abs{\lvert}{\rvert}%
\DeclarePairedDelimiter\norm{\lVert}{\rVert}%

\makeatletter
\let\oldabs\abs
\def\abs{\@ifstar{\oldabs}{\oldabs*}}

\let\oldnorm\norm
\def\norm{\@ifstar{\oldnorm}{\oldnorm*}}
\makeatother

% ošklivý hack k tomu, aby environment 'definitionnodot' neměl na konci . nebo :
% hodí se, když chceme Dělat něco jako „Definice (Riemannův integrál) je funkce...“
\newtheoremstyle{nontheoremstylenodot}{1em}{1em}{}{}{\bfseries}{}{.3em}{}
\theoremstyle{nontheoremstylenodot}
\newtheorem*{definitionnodot}{Definice}

\theoremstyle{theoremstyle}
\newtheorem*{theorem}{Věta}
\newtheorem*{lemma}{Tvrzení}

\titleformat{\section} {\normalfont\fontsize{16}{15}\bfseries}{\thesection}{1em}{}
\titleformat{\subsection} {\normalfont\fontsize{14}{15}\bfseries}{\thesubsection}{1em}{}
\titleformat{\subsubsection} {\normalfont\fontsize{12}{15}\bfseries}{\thesubsubsection}{1em}{}

\pagestyle{fancy}
\fancyhf{}
\rhead{Matematická analýza III}
\fancyfoot{}
\fancyfoot[R]{\thepage}

\begin{document}

\begin{titlepage}
    \begin{center}
        \vspace*{1cm}
            
        \Huge
        \textbf{Matematická analýza III}
            
        \vspace{0.5cm}
        \LARGE
        Stručné výpisky
        \\

        z materiálů p. doc. Klazara

        \vspace{5mm}
        
        Letní semestr 2020/2021
        
        \vspace{1.5cm}
            
        \textbf{Viktor Soukup}
        
        \vfill
        \flushright
        \normalsize
        Verze 0.3\\
        \today
        
    \end{center}
\end{titlepage}

Tyto poznámky jsem sepsal pro přípravu na zkoušku z přednášek pana doc. Klazara.
Neprošly zatím žádnou korekcí, budou tedy pravděpodobně obsahovat mnoho chyb.
Pokud v poznámkách najdete chybu, nebo pokud budete mít nějakou připomínku k tomu,
jak jsou psané, kontaktujte mě prosím na Discordu, nebo mi dejte pull-request na
\url{https://github.com/3O11/ma3-poznamky}. Ke většině vět jsem vynechal důkazy,
psal jsem je téměř výhradně k větám/tvrzením, která spadají k otázkám vypsaným ke zkoušce.
\clearpage

\tableofcontents
\clearpage

\subfile{Chapters/01-MetrickeProstory.tex}
\subfile{Chapters/02-Rady.tex}

% Temporary, until the appropriate files get created
\section{Komplexní analýza}
\subsection{Holomorfní funkce}
\subsection{Póly funkcí}
\subsection{Aplikace}

\section{Úvod do diferenciálních rovnic}
\subsection{Rovnice se separovanýmí proměnými}
\subsection{Lineární rovnice}
\subsection{Věta o existenci}

\vfill
\begin{center}
\LARGE
\textbf{The End}
\end{center}

\end{document}
