\documentclass[../main.tex]{subfiles}

\begin{document}
    
\section{Řady}

\subsection{Definice}

\begin{definition}[Řada, konvergence a divergence řady]
    Řada $\sum a_n = \sum_{n=1}^{\infty} a_n$
    je posloupnost $(a_n) \subset \mathbb{R}$, které je přiřazena
    posloupnost \underline{částečných součtů} \[ (s_n) := (a_1 + \dots + a_n) \subset \mathbb{R}. \]
    Pokud posloupnost $(s_n)$ má limitu, řekneme, že řada \underline{má součet}.
    Je-li tato limita vlastní($\in \mathbb{R}$), pak řada \underline{konverguje},
    jinak(součet je $\pm \infty$ nebo neexistuje) \underline{diverguje}.
    Součet řady se označuje stejným symbolem jako řada sama, takže také
    \[ \sum a_n = \sum_{n=1}^{\infty} a_n := \lim s_n = \lim (a_1 + \dots + a_n). \]
\end{definition}

\begin{lemma}[Nutná podmínka konvergence]
    Když řada $\sum a_n$ konverguje, pak $\lim a_n = 0$.
\end{lemma}

% This deserves some reformatting, but I'm not sure how I'd improve it.

\begin{lemma}[Harmonická řada]
    \[ \sum \frac{1}{n} = +\infty \]
\end{lemma}

\begin{lemma}
    \[ \sum \frac{1}{(n+1)n} = \frac{1}{n^2} = 1 \]
\end{lemma}

\begin{lemma}[Geometrická řada]
    Pro každé $q \in (-1,1)$ je \[ \sum_{n=1}^{\infty} q^n = \frac{1}{1-q} \]
\end{lemma}

\begin{lemma}[Leibnizovo kritérium]
    Když $a_1 \geq a_2 \geq \dots \geq 0$ a $\lim a_n = 0$, pak řada $\sum (-1)^{n-1}a_n = a_1 - a_2 + a_3 - \dots$ konverguje.
\end{lemma}

\subsection{Fourierova řada funkce}

\begin{definition}[Trigonometrická řada]
    Trigonometrická řada je řada \[ \frac{a_0}{2} + \sum_{n=1}^{\infty} (a_n\cos(nx) + b_n\sin(nx)), \]
    kde $a_n,b_n$ jsou její \underline{koeficienty} a $x\in \mathbb{R}$ je proměnná.
\end{definition}

\noindent
Trigonometrická řada je fakticky parametrický systém řad parametrizovaný proměnnou $x$.
Chceme odvodit vyjádření široké třídy funkcí $f:[-\pi,\pi]\to \mathbb{R}$,
pomocí trigonometrických řad.

\begin{definition}[Skoro skalární součin]
    Nechť $\mathcal{R}(-\pi,\pi)$ je množina všech funkcí $f:[-\pi,\pi]\to \mathbb{R}$, které
    mají na $[-\pi,\pi]$ Riemannův integrál. Pro $f,g \in \mathcal{R}(-\pi,\pi)$ definujeme
    \[ \left< f,g \right> := \int_{-\pi}^{\pi} fg \in \mathbb{R}.\footnote{Z teorie Riemannova integrálu plyne, že pokud $f,g \in \mathcal{R}(-\pi,\pi)$, pak i $fg \in \mathcal{R}(-\pi,\pi)$.} \]
\end{definition}

\noindent
Pro tento skoro skalární součin platí následující
\begin{lemma}[Symetrie, nezápornost a linearita skoro skalárního součinu]
    {\color{white} x}
    \begin{enumerate}
        \item $\left< f,g \right> = \left< g,f \right>$
        \item $\left< f,f \right> = \geq 0$
        \item $\left< af + bg, h \right> = a\left< f,h \right> + b\left< g,h \right>$
    \end{enumerate}
\end{lemma}
\noindent
ale
\begin{lemma}
    Ekvivalence
    $\left< f,f \right> = 0 \iff f \equiv 0 $
    neplatí.
\end{lemma}

\begin{definition}[$2\pi$-periodická funkce]
    Funkce je $2\pi$-periodická, když pro každé $x \in \mathbb{R}$ je $f(x + 2\pi) = f(x)$.
\end{definition}

\begin{lemma}[Ortogonalita sinů a cosinů]
    Pro každá dvě celá čísla $m,n\geq 0$ je \[ \left< \sin(mx),\cos(nx) \right> = 0. \]
    Pro každá dvě delá čísla $m,n\geq 0$, kromě $m=n=0$, je
    \[ \left< \sin(mx), \sin(nx) \right> = \left< \cos(mx), \cos(nx) \right> =
    \begin{cases} \pi \hspace{2mm} \dots \hspace{2mm} m=n\\ 0 \hspace{2mm} \dots \hspace{2mm} m\neq n. \end{cases} \]
    Konečně
    \[ \left< \sin(0x), \sin(0x) \right> = 0 \hspace{5mm} a \hspace{5mm} \left< \cos(0x), \cos(0x) \right> = 2\pi. \]
\end{lemma}

\begin{definition}[Kosinové a sinové Fourierovy koeficienty]
    Pro každou funkci $f \in \mathcal{R}(-\pi,\pi)$ definujeme její
    \underline{kosinové} Fourierovy koeficienty
    \[ a_n := \frac{\left< f(x), \cos(nx) \right>}{\pi} = \frac{1}{\pi}\int_{-\pi}^{\pi}f(x)\cos(nx)\,\text{dx}, n = 0,1,\dots \]
    a \underline{sinové} Fourierovy koeficienty
    \[ b_n := \frac{\left< f(x), \sin(nx) \right>}{\pi} = \frac{1}{\pi}\int_{-\pi}^{\pi}f(x)\sin(nx)\,\text{dx}, n = 1,2,\dots \]
\end{definition}

\begin{definition}[Fourierova řada funkce]
    Fourierova řada funkce $f$($\in \mathcal{R}(-\pi,\pi)$) je trigonometrická řada
    \[ F_f(x) := \frac{a_0}{2} + \sum_{n=1}^{\infty} \left( a_n \cos(nx) + b_n \sin(nx) \right), \]
    kde $a_n$ a $b_n$ jsou po řadě její kosinové a sinové Fourierovy koeficienty.
\end{definition}

\noindent
Geometricky nahlíženo, pracujeme v nekonečně rozměrném vektorovém prostoru se (skoro) skalárním součinem $\left<\cdot,\cdot\right>$,
v němž jsou \uv{souřadnými osami}(prvky ortogonální báze) funkce
\[ \{ \cos(nx) \mid n\in \mathbb{N}_0 \} \cup \{ \cos(nx) \mid n\in \mathbb{N} \} \]
V kontrastu s kartézskými souřadnicemi bodů v $\mathbb{R}^n$ se ale zdaleka ne každá funkce rovná součtu
své Fourierovy řady.

\begin{theorem}[Besselova nerovnost]
    Pro Fourierovy koeficienty $a_n$ a $b_n$ funkce $f \in \mathcal{R}(-\pi,\pi)$ platí nerovnost
    \[ \frac{a_0^2}{2} + \sum_{n=1}^{\infty}(a_n^2+b_n^2) \leq \frac{\left<f,f\right>}{\pi} = \frac{1}{\pi}\int_{-\pi}^{\pi}f^2. \]
\end{theorem}
\begin{proof}
    \LARGE
    \textbf{TODO}
\end{proof}

\begin{lemma}[Riemannovo-Lebesgueovo lemma]
    Pro každou funkci $f \in \mathcal{R}(-\pi,\pi)$ je\footnote{Lze dokázat pomocí Besselovy nerovnosti.}
    \[ \lim_{n\to\infty} \int_{-\pi}^{\pi}f(x)\sin(nx)\,\text{dx} = \lim_{n\to\infty} \int_{-\pi}^{\pi}f(x)\cos(nx)\,\text{dx} = 0 \]
\end{lemma}

\begin{definition}[Po částech hladká funkce]
    Funkce $f: [a,b] \to \mathbb{R}$, kde $a<b$ jsou reálná čísla, je po částech hladká, když
    existuje takové dělení
    \[ a = a_0 < a_1 < a_2 < \dots < a_k = b, k \in \mathbb{N}, \]
    intervalu $[a,b]$, že na každém intervalu $a_{i-1}, a_i, i = 1,2,\dots,k$, má
    spojitou derivaci $f'$ a pro každé $i=1,2,\dots,k$ existují vlastní jednostranné limity
    \[ f(a_i - 0) := \lim_{x\to a_i^-}f(x) \hspace{3mm} \text{a} \hspace{3mm} f'(a_i - 0) := \lim_{x\to a_i^-}f'(x) \]
    a pro každé $i=0,1,\dots,k-1$ existují vlastní jednostranné limity
    \[ f(a_i + 0) := \lim_{x\to a_i^+}f(x) \hspace{3mm} \text{a} \hspace{3mm} f'(a_i - 0) := \lim_{x\to a_i^+}f'(x) \]
\end{definition}

\noindent
Po částech hladká funkce tedy může být v několika bodech intervalu $[a,b]$ nespojitá,
ale v bodech nespojitosti má vlastní jednostranné limity a má v nich definované jednostranné nesvislé tečny.

\begin{lemma}[O Dirichletově jádře]
    Nechť $n \in \mathbb{N}$ a
    \[ J_n(x) := \frac{1}{2} + \cos(x) + cos(2x) + \dots + \cos(nx). \]
    Pak pro každé $x \in \mathbb{R} \backslash \{ 2k\pi \mid k \in \mathbb{Z} \}$
    máme \[ J_n(x) = \frac{\sin\left(\left(n + \frac{1}{2}\right)x\right)}{2\sin\left(\frac{x}{2}\right)}. \]
    také
    \[ \frac{1}{\pi}\int_{-\pi}^{0} J_n(x)\,\text{dx} = \frac{1}{\pi}\int_{0}^{\pi} J_n(x)\,\text{dx} = \frac{1}{2}. \] 
\end{lemma}

\begin{theorem}[Dirichletova]
    Nechť $f:\mathbb{R}\to\mathbb{R}$ je taková $2\pi$-periodická funkce, že její zúžení na interval $[-\pi,\pi]$
    je po částech hladké. Pak její Fourierova řada $F_f(x)$ má pro každé $a\in\mathbb{R}$ součet
    \[ F_f(a) = \frac{f(a+0) + f(a-0)}{2} = \frac{\lim_{x\to a_i^+}f(x) + \lim_{x\to a_i^-}f(x)}{2} \]
    V každém bodu spojitosti $a\in\mathbb{R}$ funkce $f(x)$ tedy její Fourierova řada má součet rovný
    funkční hodnotě, $F_f(a) = f(a)$.
\end{theorem}

\begin{definition}[Hladká funkce]
    Řekneme, že funkce $f:[a,b]\to \mathbb{R}$ je hladká, když má na intervalu $(a,b)$ spojitou derivaci $f'$ a v krajních
    bodech $a$ a $b$ mají $f(x)$ a $f'(x)$ vlastní jednostranné limity.
\end{definition}

\begin{consequence}[O hladké funkci]
    Nechť $f:\mathbb{R}\to\mathbb{R}$ je $2\pi$-periodická a spojitá funkce, jejíž
    zůžení na interval $[-\pi,\pi]$ je hladké. Potom pro každé $a\in\mathbb{R}$ je
    \[ F_f(a) = f(a). \]
    Spojitá a hladká funkce se tedy rovná součtu své Fourierovy řady.
\end{consequence}

\subsection{Basilejský problém}

\begin{example}[Basilejský problém]
    \LARGE
    TODO
\end{example}

\subsection{Divergentní řady}

Řadě $\sum a_n$, to jest posloupnosti $(a_n)\subset \mathbb{R}$, lze přiřadit její \uv{součet} i mnoha jinými způsoby,
než jen jako limitu \[ \lim s_n = \lim (a_1 + \dots + a_n) \] posloupnosti částečných součtů.
Jako ilustrace jsou uvedeny dvě sumační metody.

\begin{fact}[Abelovský součet]
    \[\lim_{x\to 1^-} \sum_{n=0}^{\infty} a_nx^n = s \Rightarrow \sum_{n=0}^{\infty} a_n \hspace{1mm} \text{\uv{=}} \hspace{1mm} s. \]
\end{fact}

\begin{fact}[Cesàrovský součet]
    \[ \lim_{n\to \infty} \frac{s_1 + s_2 + \dots + s_n}{n} = s \Rightarrow \sum_{n=0}^{\infty} a_n \hspace{1mm} \text{\uv{=}} \hspace{1mm} s \]
\end{fact}

% TODO: Add examples

\subsection{Konvergence řad}

\subsubsection{Absolutní konvergence}

\begin{definition}[Absolutní konvergence]
    Řekneme, že řada $\sum a_n = \sum_{n=1}^{\infty} a_n$ absolutně konverguje(je to absolutně konvergentní řada), pokud konverguje
    řada $\sum \abs{a_n} = \sum_{n=1}^{\infty} \abs{a_n}$ tedy
    \[ \sum_{n=1}^{\infty} \abs{a_n} < +\infty. \]
\end{definition}

\begin{definition}[Obecná absolutní konvergence]
    Nechť $A$ je nekonečná spočetná množina.\\ Pak řadou $\sum_{x\in A} a_x$(na $A$) budeme rozumět každou funkci $a:A\to \mathbb{R}$,
    kde pro $x \in A$ místo $a(x)$ stále píšeme $a_x$. Řekneme, že tato řada je obecná absolutně konvergentní řada, když
    \[ \exists c>0 \hspace{1mm} \forall \hspace{1mm} \text{konečnou množinu} \hspace{1mm} B \subset A: \sum_{x\in B}\abs{a_x} < c \]
\end{definition}

\begin{theorem}[O absolutně konvergentních řadách]
    Nechť $\sum_{x\in A} a_x$ je řada an $A$. Pak $\sum_{x\in A}a_x$ je obecná absolutně konvergentní řada, právě když
    pro libovolnou bijekci $\pi:\mathbb{N}\to A$ je klasická řada
    \[ B(\pi) = \sum_{n=1}^{\infty}b(\pi)_n = \sum_{n=1}^{\infty}b_n, \hspace{3mm} b := a_{\pi(n)}, \]
    absolutně konvergentní řada. Všechny řady $B(\pi)$ jsou pak absolutně konvergentní a mají týž součet, nezávislý na bijekci $\pi$.
\end{theorem}

\begin{definition}[Součet obecné absolutně konvergentní řady]
    Pro obecnou absolutně konvergentní řadu $\sum_{x\in A}a_x$ tak definujeme její součet jako součet $\sum b_n$ řady $\sum b_n$ s $b_n := a_{\pi(n)}$
    pro libovolnou bijekci $\pi:\mathbb{N}\to A$.
\end{definition}

\begin{definition}[Součin řad]
    Buďte $\sum_{x\in A} a_x$ a $\sum_{x\in B} b_x$ dvě obecné řady. Jejich součin, či součinová řada,
    je řada \[ \sum_{(a,b)\in A\times B} a_xb_x. \]
\end{definition}

\begin{theorem}[Součin absolutně konvergentních řad]
    Nechť $\sum_{x\in A} a_x$ a $\sum_{x\in B} b_x$ jsou obecné absolutně konvergentní řady se součty
    \[ r := \sum_{x\in A} a_x \in \mathbb{R} \hspace{3mm} a \hspace{3mm} \sum_{y\in B} b_y \in \mathbb{R}. \]
    Pak i jejich součin je obecná absolutně konvergentní řada, která má součet $rs$.
\end{theorem}

\begin{lemma}[Exponenciála]
    Pro $x\in \mathbb{R}$ nechť \[ \exp(x) := \sum_{n=0}^{\infty} \frac{x^n}{n!} = 1 + x + \frac{x^2}{2} + \frac{x^3}{6} + \frac{x^4}{24} + \dots \]
    Pak pro každé $x,y \in \mathbb{R}$ platí identita \[ \exp(x+y) = \exp(x)\exp(y). \]
\end{lemma}

\begin{lemma}[Prvočísel je $\infty$ mnoho]
    Množina prvočísel
    \[ \mathbb{P} := \{ 2,3,5,7,11,13,17,19,23, \dots \} \]
    je nekonečná.
\end{lemma}

\subsubsection{Stejnoměrná konvergence}

\subsection{Mocninné řady}

\subsection{Funkční řady}

\end{document}
