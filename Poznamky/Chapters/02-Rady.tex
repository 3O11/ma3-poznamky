\documentclass[../main.tex]{subfiles}

\begin{document}
    
\section{Řady}

\subsection{Definice}

\begin{definition}[Řada, konvergence a divergence řady]
    Řada $\sum a_n = \sum_{n=1}^{\infty} a_n$
    je posloupnost $(a_n) \subset \mathbb{R}$, které je přiřazena
    posloupnost \underline{částečných součtů} \[ (s_n) := (a_1 + \dots + a_n) \subset \mathbb{R}. \]
    Pokud posloupnost $(s_n)$ má limitu, řekneme, že řada \underline{má součet}.
    Je-li tato limita vlastní($\in \mathbb{R}$), pak řada \underline{konverguje},
    jinak(součet je $\pm \infty$ nebo neexistuje) \underline{diverguje}.
    Součet řady se označuje stejným symbolem jako řada sama, takže také
    \[ \sum a_n = \sum_{n=1}^{\infty} a_n := \lim s_n = \lim (a_1 + \dots + a_n). \]
\end{definition}

\begin{lemma}[Nutná podmínka konvergence]
    Když řada $\sum a_n$ konverguje, pak $\lim a_n = 0$.
\end{lemma}

% This deserves some reformatting, but I'm not sure how I'd improve it.

\begin{lemma}[Harmonická řada]
    \[ \sum \frac{1}{n} = +\infty \]
\end{lemma}

\begin{lemma}
    \[ \sum \frac{1}{(n+1)n} = \frac{1}{n^2} = 1 \]
\end{lemma}

\begin{lemma}[Geometrická řada]
    Pro každé $q \in (-1,1)$ je \[ \sum_{n=1}^{\infty} q^n = \frac{1}{1-q} \]
\end{lemma}

\begin{lemma}[Leibnizovo kritérium]
    Když $a_1 \geq a_2 \geq \dots \geq 0$ a $\lim a_n = 0$, pak řada $\sum (-1)^{n-1}a_n = a_1 - a_2 + a_3 - \dots$ konverguje.
\end{lemma}

\subsection{Fourierova řada funkce}

\begin{definition}[Trigonometrická řada]
    Trigonometrická řada je řada \[ \frac{a_0}{2} + \sum_{n=1}^{\infty} (a_n\cos(nx) + b_n\sin(nx)), \]
    kde $a_n,b_n$ jsou její \underline{koeficienty} a $x\in \mathbb{R}$ je proměnná.
\end{definition}

\noindent
Trigonometrická řada je fakticky parametrický systém řad parametrizovaný proměnnou $x$.
Chceme odvodit vyjádření široké třídy funkcí $f:[-\pi,\pi]\to \mathbb{R}$,
pomocí trigonometrických řad.

\begin{definition}[Skoro skalární součin]
    Nechť $\mathcal{R}(-\pi,\pi)$ je množina všech funkcí $f:[-\pi,\pi]\to \mathbb{R}$, které
    mají na $[-\pi,\pi]$ Riemannův integrál. Pro $f,g \in \mathcal{R}(-\pi,\pi)$ definujeme
    \[ \left< f,g \right> := \int_{-\pi}^{\pi} fg \in \mathbb{R}.\footnote{Z teorie Riemannova integrálu plyne, že pokud $f,g \in \mathcal{R}(-\pi,\pi)$, pak i $fg \in \mathcal{R}(-\pi,\pi)$.} \]
\end{definition}

\noindent
Pro tento skoro skalární součin platí následující
\begin{lemma}[Symetrie, nezápornost a linearita skoro skalárního součinu]
    {\color{white} x}
    \begin{enumerate}
        \item $\left< f,g \right> = \left< g,f \right>$
        \item $\left< f,f \right> = \geq 0$
        \item $\left< af + bg, h \right> = a\left< f,h \right> + b\left< g,h \right>$
    \end{enumerate}
\end{lemma}
\noindent
ale
\begin{lemma}
    Ekvivalence
    $\left< f,f \right> = 0 \iff f \equiv 0 $
    neplatí.
\end{lemma}

\begin{definition}[$2\pi$-periodická funkce]
    Funkce je $2\pi$-periodická, když pro každé $x \in \mathbb{R}$ je $f(x + 2\pi) = f(x)$.
\end{definition}

\begin{lemma}[Ortogonalita sinů a cosinů]
    Pro každá dvě celá čísla $m,n\geq 0$ je \[ \left< \sin(mx),\cos(nx) \right> = 0. \]
    Pro každá dvě delá čísla $m,n\geq 0$, kromě $m=n=0$, je
    \[ \left< \sin(mx), \sin(nx) \right> = \left< \cos(mx), \cos(nx) \right> =
    \begin{cases} \pi \hspace{2mm} \dots \hspace{2mm} m=n\\ 0 \hspace{2mm} \dots \hspace{2mm} m\neq n. \end{cases} \]
    Konečně
    \[ \left< \sin(0x), \sin(0x) \right> = 0 \hspace{5mm} a \hspace{5mm} \left< \cos(0x), \cos(0x) \right> = 2\pi. \]
\end{lemma}

\begin{definition}[Kosinové a sinové Fourierovy koeficienty]
    Pro každou funkci $f \in \mathcal{R}(-\pi,\pi)$ definujeme její
    \underline{kosinové} Fourierovy koeficienty
    \[ a_n := \frac{\left< f(x), \cos(nx) \right>}{\pi} = \frac{1}{\pi}\int_{-\pi}^{\pi}f(x)\cos(nx)\,\text{dx}, n = 0,1,\dots \]
    a \underline{sinové} Fourierovy koeficienty
    \[ b_n := \frac{\left< f(x), \sin(nx) \right>}{\pi} = \frac{1}{\pi}\int_{-\pi}^{\pi}f(x)\sin(nx)\,\text{dx}, n = 1,2,\dots \]
\end{definition}

\begin{definition}[Fourierova řada funkce]
    Fourierova řada funkce $f$($\in \mathcal{R}(-\pi,\pi)$) je trigonometrická řada
    \[ F_f(x) := \frac{a_0}{2} + \sum_{n=1}^{\infty} \left( a_n \cos(nx) + b_n \sin(nx) \right), \]
    kde $a_n$ a $b_n$ jsou po řadě její kosinové a sinové Fourierovy koeficienty.
\end{definition}

\noindent
Geometricky nahlíženo, pracujeme v nekonečně rozměrném vektorovém prostoru se (skoro) skalárním součinem $\left<\cdot,\cdot\right>$,
v němž jsou \uv{souřadnými osami}(prvky ortogonální báze) funkce
\[ \{ \cos(nx) \mid n\in \mathbb{N}_0 \} \cup \{ \cos(nx) \mid n\in \mathbb{N} \} \]
V kontrastu s kartézskými souřadnicemi bodů v $\mathbb{R}^n$ se ale zdaleka ne každá funkce rovná součtu
své Fourierovy řady.

\begin{theorem}[Besselova nerovnost]
    Pro Fourierovy koeficienty $a_n$ a $b_n$ funkce $f \in \mathcal{R}(-\pi,\pi)$ platí nerovnost
    \[ \frac{a_0^2}{2} + \sum_{n=1}^{\infty}(a_n^2+b_n^2) \leq \frac{\left<f,f\right>}{\pi} = \frac{1}{\pi}\int_{-\pi}^{\pi}f^2. \]
\end{theorem}
\begin{proof}
    \LARGE
    \textbf{TODO}
\end{proof}

\begin{lemma}[Riemannovo-Lebesgueovo lemma]
    Pro každou funkci $f \in \mathcal{R}(-\pi,\pi)$ je\footnote{Lze dokázat pomocí Besselovy nerovnosti.}
    \[ \lim_{n\to\infty} \int_{-\pi}^{\pi}f(x)\sin(nx)\,\text{dx} = \lim_{n\to\infty} \int_{-\pi}^{\pi}f(x)\cos(nx)\,\text{dx} = 0 \]
\end{lemma}

\begin{definition}[Po částech hladká funkce]
    Funkce $f: [a,b] \to \mathbb{R}$, kde $a<b$ jsou reálná čísla, je po částech hladká, když
    existuje takové dělení
    \[ a = a_0 < a_1 < a_2 < \dots < a_k = b, k \in \mathbb{N}, \]
    intervalu $[a,b]$, že na každém intervalu $a_{i-1}, a_i, i = 1,2,\dots,k$, má
    spojitou derivaci $f'$ a pro každé $i=1,2,\dots,k$ existují vlastní jednostranné limity
    \[ f(a_i - 0) := \lim_{x\to a_i^-}f(x) \hspace{3mm} \text{a} \hspace{3mm} f'(a_i - 0) := \lim_{x\to a_i^-}f'(x) \]
    a pro každé $i=0,1,\dots,k-1$ existují vlastní jednostranné limity
    \[ f(a_i + 0) := \lim_{x\to a_i^+}f(x) \hspace{3mm} \text{a} \hspace{3mm} f'(a_i - 0) := \lim_{x\to a_i^+}f'(x) \]
\end{definition}

\noindent
Po částech hladká funkce tedy může být v několika bodech intervalu $[a,b]$ nespojitá,
ale v bodech nespojitosti má vlastní jednostranné limity a má v nich definované jednostranné nesvislé tečny.

\begin{lemma}[O Dirichletově jádře]
    Nechť $n \in \mathbb{N}$ a
    \[ J_n(x) := \frac{1}{2} + \cos(x) + cos(2x) + \dots + \cos(nx). \]
    Pak pro každé $x \in \mathbb{R} \backslash \{ 2k\pi \mid k \in \mathbb{Z} \}$
    máme \[ J_n(x) = \frac{\sin\left(\left(n + \frac{1}{2}\right)x\right)}{2\sin\left(\frac{x}{2}\right)}. \]
    také
    \[ \frac{1}{\pi}\int_{-\pi}^{0} J_n(x)\,\text{dx} = \frac{1}{\pi}\int_{0}^{\pi} J_n(x)\,\text{dx} = \frac{1}{2}. \] 
\end{lemma}

\begin{theorem}[Dirichletova]
    Nechť $f:\mathbb{R}\to\mathbb{R}$ je taková $2\pi$-periodická funkce, že její zúžení na interval $[-\pi,\pi]$
    je po částech hladké. Pak její Fourierova řada $F_f(x)$ má pro každé $a\in\mathbb{R}$ součet
    \[ F_f(a) = \frac{f(a+0) + f(a-0)}{2} = \frac{\lim_{x\to a_i^+}f(x) + \lim_{x\to a_i^-}f(x)}{2} \]
    V každém bodu spojitosti $a\in\mathbb{R}$ funkce $f(x)$ tedy její Fourierova řada má součet rovný
    funkční hodnotě, $F_f(a) = f(a)$.
\end{theorem}

\begin{definition}[Hladká funkce]
    Řekneme, že funkce $f:[a,b]\to \mathbb{R}$ je hladká, když má na intervalu $(a,b)$ spojitou derivaci $f'$ a v krajních
    bodech $a$ a $b$ mají $f(x)$ a $f'(x)$ vlastní jednostranné limity.
\end{definition}

\begin{consequence}[O hladké funkci]
    Nechť $f:\mathbb{R}\to\mathbb{R}$ je $2\pi$-periodická a spojitá funkce, jejíž
    zůžení na interval $[-\pi,\pi]$ je hladké. Potom pro každé $a\in\mathbb{R}$ je
    \[ F_f(a) = f(a). \]
    Spojitá a hladká funkce se tedy rovná součtu své Fourierovy řady.
\end{consequence}

\subsection{Basilejský problém}

\begin{example}[Basilejský problém]
    \LARGE
    TODO
\end{example}

\subsection{Divergentní řady}

Řadě $\sum a_n$, to jest posloupnosti $(a_n)\subset \mathbb{R}$, lze přiřadit její \uv{součet} i mnoha jinými způsoby,
než jen jako limitu \[ \lim s_n = \lim (a_1 + \dots + a_n) \] posloupnosti částečných součtů.
Jako ilustrace jsou uvedeny dvě sumační metody.

\begin{fact}[Abelovský součet]
    \[\lim_{x\to 1^-} \sum_{n=0}^{\infty} a_nx^n = s \Rightarrow \sum_{n=0}^{\infty} a_n \hspace{1mm} \text{\uv{=}} \hspace{1mm} s. \]
\end{fact}

\begin{fact}[Cesàrovský součet]
    \[ \lim_{n\to \infty} \frac{s_1 + s_2 + \dots + s_n}{n} = s \Rightarrow \sum_{n=0}^{\infty} a_n \hspace{1mm} \text{\uv{=}} \hspace{1mm} s \]
\end{fact}

% TODO: Add examples

\subsection{Konvergence řad}

\subsubsection{Absolutní konvergence}

\begin{definition}[Absolutní konvergence]
    Řekneme, že řada $\sum a_n = \sum_{n=1}^{\infty} a_n$ absolutně konverguje(je to absolutně konvergentní řada), pokud konverguje
    řada $\sum \abs{a_n} = \sum_{n=1}^{\infty} \abs{a_n}$ tedy
    \[ \sum_{n=1}^{\infty} \abs{a_n} < +\infty. \]
\end{definition}

\begin{definition}[Obecná absolutní konvergence]
    Nechť $A$ je nekonečná spočetná množina.\\ Pak řadou $\sum_{x\in A} a_x$(na $A$) budeme rozumět každou funkci $a:A\to \mathbb{R}$,
    kde pro $x \in A$ místo $a(x)$ stále píšeme $a_x$. Řekneme, že tato řada je obecná absolutně konvergentní řada, když
    \[ \exists c>0 \hspace{1mm} \forall \hspace{1mm} \text{konečnou množinu} \hspace{1mm} B \subset A: \sum_{x\in B}\abs{a_x} < c \]
\end{definition}

\begin{theorem}[O absolutně konvergentních řadách]
    Nechť $\sum_{x\in A} a_x$ je řada an $A$. Pak $\sum_{x\in A}a_x$ je obecná absolutně konvergentní řada, právě když
    pro libovolnou bijekci $\pi:\mathbb{N}\to A$ je klasická řada
    \[ B(\pi) = \sum_{n=1}^{\infty}b(\pi)_n = \sum_{n=1}^{\infty}b_n, \hspace{3mm} b := a_{\pi(n)}, \]
    absolutně konvergentní řada. Všechny řady $B(\pi)$ jsou pak absolutně konvergentní a mají týž součet, nezávislý na bijekci $\pi$.
\end{theorem}

\begin{definition}[Součet obecné absolutně konvergentní řady]
    Pro obecnou absolutně konvergentní řadu $\sum_{x\in A}a_x$ tak definujeme její součet jako součet $\sum b_n$ řady $\sum b_n$ s $b_n := a_{\pi(n)}$
    pro libovolnou bijekci $\pi:\mathbb{N}\to A$.
\end{definition}

\begin{definition}[Součin řad]
    Buďte $\sum_{x\in A} a_x$ a $\sum_{x\in B} b_x$ dvě obecné řady. Jejich součin, či součinová řada,
    je řada \[ \sum_{(a,b)\in A\times B} a_xb_x. \]
\end{definition}

\begin{theorem}[Součin absolutně konvergentních řad]
    Nechť $\sum_{x\in A} a_x$ a $\sum_{x\in B} b_x$ jsou obecné absolutně konvergentní řady se součty
    \[ r := \sum_{x\in A} a_x \in \mathbb{R} \hspace{3mm} a \hspace{3mm} \sum_{y\in B} b_y \in \mathbb{R}. \]
    Pak i jejich součin je obecná absolutně konvergentní řada, která má součet $rs$.
\end{theorem}

\begin{lemma}[Exponenciála]
    Pro $x\in \mathbb{R}$ nechť \[ \exp(x) := \sum_{n=0}^{\infty} \frac{x^n}{n!} = 1 + x + \frac{x^2}{2} + \frac{x^3}{6} + \frac{x^4}{24} + \dots \]
    Pak pro každé $x,y \in \mathbb{R}$ platí identita \[ \exp(x+y) = \exp(x)\exp(y). \]
\end{lemma}

\begin{lemma}[Prvočísel je $\infty$ mnoho]
    Množina prvočísel
    \[ \mathbb{P} := \{ 2,3,5,7,11,13,17,19,23, \dots \} \]
    je nekonečná.
\end{lemma}

\subsubsection{Stejnoměrná a bodová konvergence}

\begin{definition}[Stejnoměrná konvergence]
    Nechť $M\subset \mathbb{R}$ je neprázdná množina a $f:M\to \mathbb{R}$ a $f_n:M\to \mathbb{R}, n = 1,2,\dots$,
    jsou na ní definované funkce. Řekneme, že $f_n$ konvergují (na $M$) stejnoměrně k $f$, symbolicky
    \[ f_n \rightrightarrows f \hspace{3mm} (\text{na} \hspace{1mm} M) \]
    když $(\varepsilon > 0)$
    \[ \forall \varepsilon \hspace{1mm} \exists n_0 = n_0(\varepsilon) \hspace{1mm} \forall x \in M: n \geq n_0 \Rightarrow \abs{f_n(x) - f(x)} < \varepsilon. \]
\end{definition}

\begin{definition}[Bodová konvergence]
    Nechť $M\subset \mathbb{R}$ je neprázdná množina a $f:M\to \mathbb{R}$ a $f_n:M\to \mathbb{R}, n = 1,2,\dots$,
    jsou na ní definované funkce.
    Pokud
    \[ \forall \varepsilon \hspace{1mm} \forall x \in M \hspace{1mm} \exists n_0 = n_0(\varepsilon, x): n \geq n_0 \Rightarrow \abs{f_n(x) - f(x)} < \varepsilon, \]
    řekneme, že $f_n$ konvergují (na $M$) k $f$ bodově, symbolicky
    \[ f_n \rightarrow f \hspace{3mm} (\text{na} \hspace{1mm} M) \]
\end{definition}

\noindent
Jinými slovy, $\forall x \in M: \lim f_n(x) = f(x)$.
Stejnoměrná konvergence implikuje bodovou, ale ne naopak.

\begin{definition}[Supremová norma]
    Pro funkci $f: M \to \mathbb{R}$ definujeme její supremovou normu $\norm{f}_\infty$ jako
    \[ \norm{f}_\infty := \sup(\{ \abs{f(x)} \mid x \in M \}) \in [0, +\infty], \]
    s hodnotou $+\infty$ pro shora neomezenou množinu $\{ \dots \}$.
\end{definition}

\begin{lemma}[Kritérium $\rightrightarrows$]
    Nechť $M \subset \mathbb{R}$ je neprázdná množina a $f:M\to \mathbb{R}$ a $f_n:M\to \mathbb{R}, n = 1,2,\dots$,
    jsou na ní definované funkce. Pak
    \[ f_n \rightrightarrows f \hspace{3mm} (\text{na} \hspace{1mm} M) \iff \lim_{n\to \infty} \norm{f - f_n}_\infty = 0. \]
\end{lemma}

\begin{definition}[Lokálně stejnoměrná konvergence]
    Nechť $M \subset \mathbb{R}$ je neprázdná množina a $f:M\to \mathbb{R}$ a $f_n:M\to \mathbb{R}, n = 1,2,\dots$,
    jsou na ní definované funkce.
    Lokálně stejnoměrná konvergence $f_n$k $f$ (na $M$), symbolicky $f_n \locconv f$ (na $M$),
    znamená, že
    \[ \forall a \in M \hspace{1mm} \exists \delta > 0: f_n \rightrightarrows f \hspace{1mm} (\text{na} \hspace{1mm} M \cap (a-\delta, a+\delta)). \]
\end{definition}

\begin{theorem}[$\locconv$ zachovává spojitost]
    Nechť $M \subset \mathbb{R}$, $f:M\to \mathbb{R}$, $f_n:M\to \mathbb{R}$ pro $n \in \mathbb{N}$,
    každá funkce $f_n$ je spojitá a \[ f_n \locconv f \hspace{1mm} (\text{na} \hspace{1mm} M). \]
    Pak i $f$ je spojitá.
\end{theorem}
\begin{proof}
    Nechť $a\in M$ a buď dáno $\varepsilon > 0$. Vezmeme $\delta > 0$,
    že $f_n$ konvergují na $N := M \cap (a-\delta,a+\delta)$ stejnoměrně.
    Vezmeme $n_0$, že $n\geq n_0 \land x \in N \Rightarrow \abs{f(x) - f_n(x)} < \frac{\varepsilon}{3}$.
    Vezmeme libovolné $n_1 \geq n_0$ a pak, díky spojitosti $f_{n_1}$, takové $\delta \in (0, \delta)$,
    že \[ x \in M \cap (a - \delta_0, a+\delta_0)(\subset N) \Rightarrow \abs{f_{n_1}(a) - f_{n_1}(x)} < \frac{\varepsilon}{3}. \]
    Pak pro každé $x \in M \cap (a - \delta_0, a+\delta_0)(\subset N)$ máme, že
    \[ \abs{f(a) - f(x)} \leq \abs{f(a) - f_{n_1}(a)} + \abs{f_{n_1}(a) - f_{n_1}(x)} + \abs{f_{n_1}(x) - f(x)} <
    \frac{\varepsilon}{3} + \frac{\varepsilon}{3} + \frac{\varepsilon}{3} = \varepsilon \]
    Funkce $f$ je spojitá v bodě $a$.
\end{proof}

\begin{lemma}[Weierstrassův test]
    Nechť $M \subset \mathbb{R}$ je neprázdná množina, $f:M\to \mathbb{R}$, $f_n:M\to \mathbb{R}$ a $\sum f_n \to f$ (na $M$).
    Pak \[ \sum_{n=1}^{\infty} f_n \rightrightarrows f \hspace{3mm} (\text{na} \hspace{1mm} M),
    \hspace{1mm} \text{pokud} \hspace{3mm} \sum_{n=1}^{\infty}F_n := \sum_{n=1}^{\infty} \norm{f_n}_\infty < +\infty. \]
\end{lemma}

\subsection{Mocninné řady}

\begin{definition}[Mocninná řada]
    Mocninná řada se středem $a\in\mathbb{R}$ a koeficienty $a_n \in\mathbb{R}$(a proměnnou $x\in\mathbb{R}$)
    je funkční řada \[ \sum_{n=0}^{\infty} a_n(x-a)^n. \]
\end{definition}

\begin{definition}[Poloměr konvergence]
    Poloměr konvergence $R$ mocninné řady \[ \sum_{n=0}^{\infty} a_nx^n \] je nezáporné reálné číslo nebo $+\infty$:
    \[ R := \frac{1}{\limsup \abs{a_n}^\frac{1}{n}} \in [0, +\infty], \]
    kde $\frac{1}{0} = +\infty$ a $\frac{1}{+\infty} := 0$. S těmito konvencemi máme i ekvivalentní
    vztah $\limsup \abs{a_n}^{\frac{1}{n}} = \frac{1}{R}$.
\end{definition}

\begin{theorem}[O konvergencích mocninných řad]
    Nechť \[ F(x) := \sum_{n=0}^{\infty} a_nx^n \]
    je mocninná řada s poloměrem konvergence $R$. Pak pro každé reálné $x$ s $\abs{x} < R$ řada $F(x)$ absolutně konverguje a
    pro $\abs{x} > R$ diverguje. Když $R > 0$, pak na intervalu $(-R,R)$ řada $F(x)$ konverguje lokálně stejnoměrně ke svému (bodovému) součtu.
\end{theorem}

\begin{theorem}[Počítání s mocninnými řadami]
    Nechť \[ A(x) := \sum_{n\geq 0} a_nx^n \hspace{3mm} a \hspace{3mm} B(x) := \sum_{n\geq 0} b_nx^n \]
    jsou mocninné řady konvergující na nějakém intervalu $I := (-a,a)$, kde $a > 0$. Označme stejně i odpovídající funkce $A,B:I\to \mathbb{R}$.
    Pro jejich (formální) součet, součin, podíl a derivaci platí následující.
    \begin{enumerate}
        \item {
            Mocninná řada (Formální součet)
            \[ C(x) := \sum_{n\geq 0}(a_n + b_n)x^n \]
            konverguje na $I$ a pro každé $x \in I$ je $C(x) = A(x) + B(x)$.
        }
        \item {
            Mocninná řada (Formální součin)
            \[ C(x) := \sum_{n\geq 0} \left( \sum_{k= 0}^{n}a_nb_{n-k} \right)x^n \]
            konverguje na $I$ a pro každé $x \in I$ je $C(x) = A(x) \cdot B(x)$.
        }
        \item {
            Nechť $b_0 \neq 0$ a $d_n := -\frac{b_n}{b_0}$. Pak existuje $b > 0$, že mocninná řada (Formální podíl)
            \[ C(x) = \sum_{n\geq 0}c_nx^n = \frac{A(x)}{B(x)} := \frac{1}{b_0} \sum_{n=0}^{\infty}a_nx^n\sum_{n=0}^{\infty}(d_1x + d_2x^2 + \dots)^n \]
            konverguje na intervalu $J := (-b,b)$ a pro každé $x \in J$ je $C(x) = \frac{A(x)}{B(x)}$.
        }
        \item {
            Mocninná řada (Formální derivace)
            \[ C(x) := \sum_{n\geq 1}^{\infty} na_nx^{n-1} \]
            konverguje na $I$ a pro každé $x \in I$ je $C(x) = A'(x)$.
        }
    \end{enumerate}
    Ve třetí části je použita formální geometrická řada:
    \[ \frac{1}{1-(d_1x + d_2x^2 + \dots)} = \sum_{n=0}^{\infty}(d_1x + d_2x^2 + \dots)^n. \]
\end{theorem}

\begin{lemma}[Abelova nerovnost]
    Pro $i = 1,2,\dots,n$ nechť $a_i \in \mathbb{C}, b_i \in \mathbb{R}$ s $b_1 \geq b_2 \geq \dots \geq b_n \geq 0$, $A_i := a_1 + a_2 + \dots + a_i$
    a $A[n] := \max(\abs{A_1}, \abs{A_2}, \dots, \abs{A_n})$. prostorem
    \[ \left| \sum_{i=1}^{n}a_ib_i \right| \leq A[n]\cdot b_1. \]
\end{lemma}

\begin{theorem}[Abelova]
    Nechť
    \[ A(x) := \sum_{n=0}^{\infty} a_nx^n \]
    je mocninná řada s poloměrem konvergence $R\in (0,+\infty)$ a označme stejně odpovídající
    funkci $A: (-R,R) \to\mathbb{R}$. Když řada $\sum_{n\geq 0} a_nR^n$ konverguje a má součet
    \[ S := \sum_{n\geq 0} a_nR^n, \]
    pak je limita zleva v $R$ funkce $A(x)$ rovna $S$:
    \[ \lim_{x\to R^-} A(x) = \lim_{a\to R^-} \sum_{n=0}^{\infty}a_nx^n = \sum_{n=0}^{\infty}a_nR^n = S. \]
\end{theorem}

\subsubsection{Pólyova věta o náhodných procházkách}

\begin{definition}[Graf]
    Graf $G = (V,E)$ sestává z množiny \underline{vrcholů} $V$ a množiny \underline{hran} $E \subset {{V}\choose{2}}$.
    Zde \[ {{V}\choose{2}} := \{ A \mid A \subset V \land \abs{A} = 2 \} \]
    je množina všech dvouprvkových podmnožin množiny $V$.
\end{definition}

\begin{definition}[$d$-regulární graf]
    Graf $G = (V,E)$ je $d$-regulární, $D \in\mathbb{N}$, má-li každý vrchol $d$ sousedů,
    to jest \[\forall v \in V: \abs*{\overbrace{\{ u\in V \mid \{ u,v \} \in E \}}^{N(v)}} = d. \]
\end{definition}

\begin{definition}[Lokálně konečný graf]
    Graf $G$ je lokálně konečný, má-li každý vrchol $v \in V$ jen konečně mnoho sousedů, tj. množina $N(v)$ je konečná.
\end{definition}

\begin{definition}[Procházka]
    Procházka $w$ v grafu $G = (V,E)$ je taková \underline{konečná}, $w = (v_0,v_1,\dots,v_n)$ s délkou $\abs{w} := n \in \mathbb{N}_0$,
    či \underline{nekonečná}, $w = (v_0,v_1, \dots)$, posloupnost vrcholů $v_i \in V$, že pro každé $i \in \mathbb{N}_0 (< n)$ je
    $\{v_i,v_{i+1} \in E\}$. Vrchol $v_0$ pojmenujeme jako \underline{start procházky} $w$.
\end{definition}

\begin{definition}[Počet procházek]
    Definujeme \[ d_n(v_0,G) := \abs{\{ w \mid w \subset V \hspace{1mm} \text{je procházka se startem} \hspace{1mm} v_0 \hspace{1mm} a \hspace{1mm} \abs{w} = n \}}, \]
    počet procházek v grafu $G$ s daným startem $v_0$ a s délkou $n$.
\end{definition}

\begin{definition}[Rekurentní procházka]
    Rekurentní procházka $w = (v_0,  v_1, \dots, v_n)$ opětovně prochází
    startem: existuje $i \in \{ 1,2,\dots,n \}$, že $v_i = v_0$
\end{definition}

\begin{definition}[Počet rekurentních procházek]
    Jako \[ a_n(v_0,G) := \abs{\{ w \mid w\subset V \hspace{1mm} \text{je rekurentní procházka se startem} \hspace{1mm} v_0 \hspace{1mm} a \hspace{1mm} \abs{w} = n \}} \]
    označíme počet rekurentních procházek v grafu $G$ s daným startem $v_0$ a s délkou $n$.
\end{definition}

\begin{definition}[Automorfismus]
    Automorfismus grafu $G = (V,E)$ je taková bijekce $f:V\to V$,
    že \[ \forall u,v \in V: \{ u,v \} \in E \iff \{ f(u), f(v) \in E \}. \]
\end{definition}

\begin{definition}[(Vrcholově) tranzitivní graf]
    Graf $G = (V,E)$ je (vrcholově) tranzitivní, když
    \[ \forall u,v\in V \hspace{1mm} \exists F: F \hspace{1mm} \text{je automorfismus} \hspace{1mm} G \land F(u) = v. \]
\end{definition}

\begin{lemma}[Procházky v grafech]
    Počet procházek, popř. rekurentních procházek, dané délky v tranzitivním grafu nezávisí na startu:
    když je $G = (V,E)$ tranzitivní a lokálně konečný, pak pro každé $n \in \mathbb{N}_0$ a každé dva vrcholy
    $u,v \in V$ je
    \[ d_n(u,G) = d_n(v,G), \hspace{3mm} \text{popř.} \hspace{3mm} a_n(u,G) = a_n(v,G). \]
\end{lemma}

\noindent
V tranzitivních grafech $G$ budeme stručně označovat počty procházek, resp. rekurentních procházek, s délkou $n$
jako $d_n(G)$, resp. $a_n(G)$.

\begin{example}[Nekonečná cesta]
    Nekonečná cesta \[ P = (\mathbb{Z}, \{\{n,n+1\} \mid n \in \mathbb{Z}\}) \]
    je tranzitivní a $2$-regulární.
\end{example}

\begin{definition}[Zobecněná nekonečná cesta]
    Zobecnněním nekonečné cesty je pro $d\in \mathbb{N}$ graf
    \[ \mathbb{Z}^d := \left(\mathbb{Z}^d, \left\{\left\{ \overline{u},\overline{v} \right\} \mid \sum_{i=1}^{d} \abs{u_i - v_i} = 1 \right\}\right), \]
    kde píšeme $\overline{u} = (u_1, \dots, u_d) \in \mathbb{Z}^d$.
\end{definition}

\begin{lemma}
    Grafy $\mathbb{Z}^d$ jsou tranzitivní a $2d$-regulární.
\end{lemma}

\begin{theorem}[Slabá Abelova]
    Když mocninná řada
    \[ U(x) := \sum_{n=0}^{\infty} u_nx^n \in \mathbb{R}[[x]] \]
    konverguje pro každé $x \in [0,R)$, kde $R \in (0, +\infty)$
    je reálné číslo, a má všechny koeficienty $u_n \geq 0$,
    pak následující limita a suma jsou definované a rovnají se -
    \[ \lim_{x\to R^-} U(x) = \sum_{n=0}^{\infty} u_nR^n \hspace{3mm} (=: U(R)) \]
    - bez ohledu na to, zda jsou konečné nebo $+\infty$.
\end{theorem}

\begin{theorem}[Stirlingův vzorec]
    \[ n! \sim \sqrt{2\pi n} \left( \frac{n}{e} \right)^n, \hspace{3mm} n\to \infty. \]
\end{theorem}

\begin{theorem}[Pólya]
    Pro $d = 1$ a $2$ je \[ \lim_{n\to\infty}\frac{a_n(\mathbb{Z}^d)}{d_n(\mathbb{Z}^d)} = \lim_{n\to\infty}\frac{a_n(\mathbb{Z}^d)}{(2d)^n} = 1 \]
    a pro $d \geq 3$ je
    \[ \lim_{n\to\infty} \frac{a_n(\mathbb{Z}^d)}{d_n(\mathbb{Z}^d)} = \lim_{n\to\infty}\frac{a_n(\mathbb{Z}^d)}{(2d)^n} < 1 \]
\end{theorem}
\begin{proof}
    \LARGE
    \textbf{TODO}
\end{proof}

\begin{proof}
    \LARGE
    \textbf{TODO}
\end{proof}

\end{document}
