\documentclass[11pt]{article}

\usepackage[czech]{babel}
\usepackage{a4wide}
\usepackage[utf8]{inputenc}
\usepackage[T1]{fontenc}
\usepackage{fancyhdr}
\usepackage{amssymb}
\usepackage{amsthm}
\usepackage{amsmath}
\usepackage{mathtools}
\usepackage{mleftright}
\usepackage{subfig}

% forcing footnotes to be at the very bottom
\usepackage[bottom]{footmisc}

\usepackage{hyperref}
\usepackage{titlesec}
%This has to be the last
\usepackage{subfiles}

\usepackage{geometry}
\geometry{
    a4paper,
    total={170mm,257mm},
    right=20mm,
    left=20mm,
    top=30mm,
    bottom=20mm,
}

\DeclareMathOperator{\rank}{rank}
\DeclareMathOperator{\Span}{span}

\newtheoremstyle{nontheoremstyle}{1em}{1em}{}{}{\bfseries}{:}{.5em}{}
\newtheoremstyle{theoremstyle}{1em}{1em}{\it}{}{\bfseries}{:}{.5em}{}

\theoremstyle{nontheoremstyle}
\newtheorem*{definition}{Definice}
\newtheorem*{example}{Příklad}
\renewenvironment{proof}{{\noindent\bfseries Důkaz:}}{\qed}
\newtheorem*{intuition}{Intuice}
\newtheorem*{remark}{Poznámka}
\newtheorem*{consequence}{Důsledek}
\newtheorem*{observation}{Pozorování}

% Absolutni hodnota a norma
\DeclarePairedDelimiter\abs{\lvert}{\rvert}%
\DeclarePairedDelimiter\norm{\lVert}{\rVert}%

\makeatletter
\let\oldabs\abs
\def\abs{\@ifstar{\oldabs}{\oldabs*}}

\let\oldnorm\norm
\def\norm{\@ifstar{\oldnorm}{\oldnorm*}}
\makeatother

% ošklivý hack k tomu, aby environment 'definitionnodot' neměl na konci . nebo :
% hodí se, když chceme Dělat něco jako „Definice (Riemannův integrál) je funkce...“
\newtheoremstyle{nontheoremstylenodot}{1em}{1em}{}{}{\bfseries}{}{.3em}{}
\theoremstyle{nontheoremstylenodot}
\newtheorem*{definitionnodot}{Definice}

\theoremstyle{theoremstyle}
\newtheorem*{theorem}{Věta}
\newtheorem*{lemma}{Tvrzení}

\titleformat{\section} {\normalfont\fontsize{16}{15}\bfseries}{\thesection}{1em}{}
\titleformat{\subsection} {\normalfont\fontsize{14}{15}\bfseries}{\thesubsection}{1em}{}
\titleformat{\subsubsection} {\normalfont\fontsize{12}{15}\bfseries}{\thesubsubsection}{1em}{}

\pagestyle{fancy}
\fancyhf{}
\rhead{Matematická analýza III}
\lhead{Otázky na Letní semestr 2020/2021}
\fancyfoot{}
\fancyfoot[R]{\thepage}

\begin{document}

\section{Definujte metrický prostor a sférickou metriku. Dokažte, že hemisféra není plochá.}

\begin{definition}[Metrický prostor]
    Metrický prostor je dvojice $(M,d)$ množiny $M\neq \emptyset$ a zobrazení \[d: M \times M \to \mathbb{R}\]
    zvaného \underline{metrika} či \underline{vzdálenost}, které $\forall x,y,z \in M$ splňuje:
    \begin{enumerate}
        \item $d(x,y) = 0 \iff x=y$
        \item $d(x,y) = d(y,x)$
        \item $d(x,y) \leq d(x,z) + d(z,y)$
    \end{enumerate}
    Z těchto podmínek plyne i $d(x,y) \geq 0$.
\end{definition}

\begin{example}[Sférická metrika]
    Jako \[S := \{ (x_1,x_2,x_3) \in \mathbb{R}^3 \mid x_1^2 + x_2^2 + x_3^3 = 1 \}\] si označíme \underline{jednotkovou sféru} v
    euklidovském prostoru $\mathbb{R}^n$. Funkci $s: S\times S \to [0,\pi]$ definujeme pro $\overline{x}, \overline{y} \in S$
    jako \[ s(\overline{x}, \overline{y}) = \begin{cases}
        0 \dots \overline{x} = \overline{y}\\
        \varphi \dots \overline{x} \neq \overline{y}
    \end{cases} \]
    kde $\varphi$ je úhel sevřený dvěma polopřimkami procházejícímí počátkem $\overline{0}$ a body $\overline{x}$ a $\overline{y}$.
    Tento úhel je vlastně délka kratšího z oblouků mezi body $\overline{x}$ a $\overline{y}$ na jednotkové kružnici vytknuté na $S$ rovinou určenou
    počátkem a body $\overline{x}$ a $\overline{y}$. Funkci $s$ nazveme sférickou metrikou.
\end{example}

\begin{theorem}[$H$ není plochá]
    Metrický prostor $(H,s)$ není izometrický žádnému Euklidovskému prostoru $(X,e_n)$ s $X \subset \mathbb{R}^n$    
\end{theorem}
\begin{proof}
    Následující vlastnost vzdáleností daných čtyřmi body $t,u,v$ a $w$ v Euklidovském prostoru $(\mathbb{R}^n, e_n)$
    není splňena v $(H,s)$:
    \[ e_n(t,u) = e_n(t,v) = e_n > 0 \land e_n(t,w) = e_n(w,u) = \frac{1}{2}e_n(t,u) \Rightarrow e_n(w,v) = \frac{\sqrt{3}}{2}e_n(t,v)\,\, (<e_n(t,v)). \]
    Podle předpokladu implikace body $t,u$ a $v$ tvoří rovnostranný trojúhelník se stranou délky $x>0$ a $w$ má od $t$ i $u$ vzdálenost $\frac{x}{2}$.
    Podle předchozího tvrzení je pak $w$ středem úsečky $tu$. Tyto čtyři body jsou tedy koplanární(leží v jedné rovině) a úsečka $vw$ je výčka spoštěná z
    vrcholu $v$ rovnostranného trojúhélníka $tuv$ na stranu $tu$. Podle Pythagorovy věty se její délka $e_2(v,w) = e_n(v,w)$ rovná $\frac{\sqrt{3}}{2}x$,
    což říká závěr implikace.

    Na hemisféře $(H,s)$ nalezneme čtyři různé body $t,u,v$ a $w$ splňující předpoklad předchozí implikace, ale ne její závěr. Z toho plyne,
    že izometrie mezi hemisférou a Euklidovským prostorem neexistuje, protože každá izometrie ze své definice implikaci zachovává.
    Tyto body jsou
    \[ t=(1,0,0),u=(0,1,0), v=(0,0,1) \,\,\text{a}\,\, w=\left(\frac{1}{\sqrt{2}}, \frac{1}{\sqrt{2}}, 0\right) . \]
    Patrně $s(t,u) = s(t,v) = s(u,v) = \frac{\pi}{2}$ a $s(t,w) = s(w,u) = \frac{1}{2}s(t,u) = \frac{\pi}{4}$.
    Bod $v$ je \uv{severní pól} ($x_3 = 0$) a $w$ je střed oblouku $tu$. Ale všechny body na rovníku mají od pólu vzdálenost $\frac{\pi}{2}$.
    Takže $s(w,v) = s(t,v)$ a závěr implikace neplatí.
\end{proof}

\section{Dokažte Ostrowskiho větu.}

\begin{definition}[$p$-adický řád]
    Nechť $p \in \{ 2,3,5,7,11,\dots \}$ je prvočíslo a nechť $n \in \mathbb{Z}$ je nenulové celé číslo.
    Jako $p$-adický řád čísla $n$ definujeme
    \[ \text{ord}_p(n) := \max(\{ m \in \mathbb{N}_0 : p^m \mid n \})\footnote{$\cdot \mid \cdot$ značí relaci dělitelnosti.} \]
    Dále ještě $\forall p$ definujeme $\text{ord}_p(0) := +\infty$.
\end{definition}

\begin{remark}[Rozšíření $\text{ord}_p(\cdot)$ na zlomky]
    Pro nenulové $\alpha = \frac{a}{b} \in \mathbb{Q}$ definujeme
    \[ \text{ord}_p(\alpha) := \text{ord}_p(a) - \text{ord}_p(b) \]
    Jinak opět $\text{ord}_p(0) = \text{ord}_p(\frac{0}{b}) := +\infty$.
\end{remark}

\begin{definition}[$p$-adická norma]
    Fixujeme reálnou konstantu $c \in (0,1)$ a definujeme funkci $\lvert \cdot \rvert_p : \mathbb{Q} \to [0, +\infty)$ jako
    \[ \left| \frac{a}{b} \right|_p := c^{\text{ord}_p\left( \frac{a}{b} \right)} \]
    kde klademe $\lvert 0 \rvert_p = c^{+\infty} := 0$
\end{definition}

\begin{definition}[Kanonická $p$-adická norma]
    Pro $\alpha \in\mathbb{Q}$ a prvočíslo $p$ je kanonická $p$-adická norma $|| \cdot ||_p$ definovaná jako
    \[|| \alpha ||_p := p^{-\text{ord}_p(\alpha)}\]
    to jest v obecné $p$-adické normě $|| \cdot ||_p$ klademe $c := \frac{1}{p}$.
\end{definition}

\begin{lemma}
    Nechť $\norm{\cdot}$ je netriviální norma na tělese $\mathbb{Q}$.
    Potom $\exists n \in \mathbb{N}: n \geq2 \land \norm{n} \neq 1.$
\end{lemma}

\begin{lemma}
    Pro každá dvě nesoudělná $a,b \in\mathbb{Z}$ existují čísla $c,d\in\mathbb{Z}$, že
    \[ ac + bd = 1 \]
\end{lemma}

\begin{theorem}[A. Ostrowski]
    Nechť $||\cdot||$ je norma na tělese racionálních čísel $\mathbb{Q}$. Pak nastává jedna ze tří následujících možností.
    \begin{enumerate}
        \item Je to triviální norma.
        \item Existuje reálné $c\in (0,1]$ takové, že $||x|| = |x|^c$.
        \item Existuje reálné $c \in (0,1)$ a prvočíslo $p$, že $||x|| = |x|_p = c^{\text{ord}_p(x)}$(kde $c^\infty := 0$).
    \end{enumerate}
    Modifikovaná absolutní hodnota a $p$-adické normy jsou tedy jediné netriviální
    normy na tělese racionálních čísel.
\end{theorem}
\begin{proof}
    Nechť $\norm{\cdot}$ je netriviální. Pak díky prvnímu z pomocných tvrzení existuje $n \in\mathbb{N}\backslash \{1\}$,
    že $\norm{n} \neq 1$. Máme dva případy.
    \begin{enumerate}
        \item {
            \textbf{Existuje $n\in\mathbb{N}$, že $\norm{n} > 1$}. Jako $n_0$ označíme nejmenší takové $n$.
            Patrně $n_0 \geq 2$ a \[ 1 \leq m < n_0 \Rightarrow \norm{m} \leq 1. \]
            Existuje jednoznačné reálné číslo $c > 0$, že
            \[ \norm{n_0} = n_0^c. \]
            Každé $n\in\mathbb{N}$ lze při základu $n_0$ zapsat jako
            \[ n = a_0 + a_1n_0 + a_2n_0^2 + \dots + a_sn_0^s, \,\,\, \text{kde} \,\,\, a_i,s\in\mathbb{N}_0, 0 \leq a_i < n_0 \,\, \text{a} \,\, a_s \neq 0. \]
            Pro $n_0 = 10$ jde o obvyklý zápis v desítkové soustavě. Takže
            \begin{align*}
                \norm{n} &= \norm{a_0 + a_1n_0 + a_2n_0^2 + \dots + a_sn_0^s}\\
                &\leq \sum_{j=0}^{s} \norm{a_j} \cdot \norm{n_0}^j\\
                &\leq \sum_{j=0}^{s} n_{0}^{js} \leq n_0^{sc} \sum_{i=0}^\infty \left( \frac{1}{n_0^c} \right)^i\\
                &\leq n^cC, \,\,\,\, \text{kde} \,\, C := \sum_{i=0}^\infty \left( \frac{1}{n_0^c} \right)^i
            \end{align*}
            Tedy \[ \forall n \in \mathbb{N}_0: \norm{n} \leq Cn^c. \]
            Tato nerovnost ve skutečnosti platí dokonce s $C = 1$. PRo každé $m,n\in\mathbb{N}$ multiplikativita normy a předchozí nerovnost
            dávají \[ \norm{n}^m = \norm{n^m} \leq C(n^m)^c = C(n^c)^m. \]
            Vezmeme-li zde $m$-tou odmocninu, dostaneme $\norm{n} \leq C^{\frac{1}{m}}n^c$. Pro
            $m\to \infty$ máme $C^{\frac{1}{m}}\to 1$.
            Takže skutečně \[\forall n \in \mathbb{N}_0: \norm{n} \leq n^c.\]

            Nyní podobně odvodíme opačnou nerovnost $\norm{n} \geq n^c, n\in\mathbb{N}_0$.
            Pro každé $n \in \mathbb{N}$ hořejší zápis čísla $n$ při základu $n_0$ dává \[ n_0^{s+1} > n \geq n_0^s. \]
            Podle $\Delta$-ové nerovnosti máme \[ \norm{n_0}^{s+1} = \norm{n_0^{s+1}} \leq \norm{n} + \norm{n_0^{s+1} - n}. \]
            Tedy
            \begin{align*}
                \norm{n} &\geq \norm{n_0}^{s+1} - \norm{n_0^{s+1} - n} \geq n_0^{(s+1)c} - (n_0^{s+1} - n)^c\\
                &\geq n_0^{(s+1)c} - (n_0^{s+1} - n_0^s)^c = n_0^{(s+1)c}\left( 1- \left( 1- \frac{1}{n_0} \right)^c \right)\\
                &\geq n^cC', \,\,\, \text{kde} \,\, C':= 1- \left(1 - \frac{1}{n_0}\right)^c > 0.
            \end{align*}
            Trik s $m$-tou odmocninou opět dává
            \[ \forall n \in \mathbb{N}_0: \norm{n}\geq n^c \]
            a tedy
            \[ \forall n \in \mathbb{N}_0: \norm{n} = n^c. \]
            Z multiplikativity normy dostáváme $\norm{x} = \abs{x}^c$ pro každý zlomek $x \in\mathbb{Q}$. Podle tvrzení výše je $c \in (0,1]$. Odvodili jsme,
            že platí případ 2 Ostrowskiho věty.
        }
        \item {
            \textbf{Zbývá případ, kdy pro každé $n\in\mathbb{N}$ je $\norm{n}\leq 1$ a existuje $n\in\mathbb{N}$, že $\norm{n}<1$.}
            Nechť $n_0$ je nejmenší takové $n$, opět $n_0 \geq 2$. Tvrdíme, že $n_0 = p$ je prvočíslo. Kdyby totiž $n_0$ mělo rozklad
            $n_0 = n_1n_2$ s $n_i \in\mathbb{Z}$ a $1 < n_1,n_2<n_0$, dostali bychom spor
            \[ 1 > \norm{n_0} = \norm{n_1n_2} = \norm{n_1}\cdot\norm{n_2} = 1 \cdot 1 = 1,\]
            kde jsme použili multiplikativitu normy a to, že $\norm{m} = 1$ pro každé $m \in\mathbb{N}$ s $1 \leq m < n_0$.
            Ukážeme, že každé jiné prvočíslo $q\neq p$ má normu $\norm{q} = 1$. Pro spor nechť $q\neq p$ je další prvočíslo s
            normou $\norm{q} < 1$. Vezmeme tak velké $m \in\mathbb{N}$, že $\norm{p}^m,\norm{q}^m < \frac{1}{2}$.
            Podle známého výsledku v elementární teorii čísel výše existují celá čísla $a$ a $b$, že $aq^m + bp^m = 1$.
            Znormování této rovnosti dává spor
            \[ 1 = \norm{1} = \norm{aq^m + bp^m} \leq \norm{a}\cdot\norm{q}^m + \norm{b}\cdot\norm{p}^m < 1 \cdot \frac{1}{2} + 1 \cdot \frac{1}{2} = 1. \]
            Zde jsme využili trojúhelníkovou nerovnost, multiplikativitu normy a to, že nyní $\norm{a} \leq 1$ pro každé $a \in\mathbb{Z}$.

            Tedy $\norm{q} = 1$ pro každé prvočíslo $q$ různé od $p$. Odtud pomocí multiplikativity normt a rozkladu nenulového zlomku
            $x$ na součin mocnin prvočísel dostáváme vyjádření
            \begin{align*} \norm{x} &= \norm{\prod_{q=2,3,5,\dots} q^{\text{ord}_q(x)}} = \prod_{q=2,3,5,\dots} \norm{q^{\text{ord}_q(x)}} = \norm{p}^{\text{ord}_p(x)}\\
            &= c^{\text{ord}_p(x)}\,\,\, \text{kde} \,\, c := \norm{p} \in (0,1). \end{align*}

            Též $\norm{0} = c^{\text{ord}_p(0)} = c^\infty = 0$. Dostali jsme případ 3 Ostrowskiho věty.
        }
    \end{enumerate}
\end{proof}

\section{Dokažte Heine-Borelovu větu.}

\begin{lemma}[Topologická spojitost]
    Nechť $f: M \to N$ je zobrazení mezi metrickými prostory $(M,d)$ a $(N,e)$. prostorem
    \[ f \hspace{1mm} \text{je spojité} \iff \forall \hspace{1mm} \text{OM} \hspace{1mm} A \subset N: f^{-1}[A] = \{ x\in M \mid f(x) \in A \} \subset M \hspace{1mm} \text{je OM}.
    \footnote{OM zkracuje sousloví \uv{otevřená množina}. } \]
\end{lemma}

\noindent
Toto tvrzení platí i pro uzavřené množiny.

\begin{lemma}[Spojitý obraz kompaktu]
    Nechť $(M,d)$ a $(N,e)$ jsou metrické prostory, $X\subset M$ je neprázdná kompaktní množina a \[ f: X\to N \] je spojitá funkce.
    Pak obraz $f[X] \subset N$ je kompaktní množina.
\end{lemma}

\begin{lemma}[Spojitost inverzu]
    Nechť $f: X\to N$ je spojité zobrazení z neprázdné kompaktní množiny $X \subset M$ v metrickém prostoru $(M,d)$ do $(N,e)$.
    Potom inverzní zobrazení \[ f^{-1}: f[X] \to X \] je spojité.
\end{lemma}

\begin{definition}[Homeomorfismus]
    Zobrazení $f: M\to N$ mezi metrickými prostory $(M,d)$ a $(N,e)$ je jejich homeomorfismus, je-li $f$ bijekce a jsou-li $f$ a $f^{-1}$ spojitá zobrazení.
    Pokud mezi $(M,d)$ a $(N,e)$ existuje homeomorfismus, jsou \underline{homeomorfní}.
\end{definition}

\begin{definition}[Topologická kompaktnost]
    Podmnožina $A\subset M$ metrického prostoru $(M,d)$ je topologicky kompaktní, pokud každý systém otevřených množin $\{ X_i \mid i \in I \}$ v $M$ platí:
    \[ \bigcup_{i \in I} X_i \supset A \Rightarrow \exists \hspace{1mm} \text{konečná množina} \hspace{1mm} J \subset I: \bigcup_{i\in J}X_i \supset A. \]
\end{definition}

\begin{theorem}[Heine-Borelova]
    Podmnožina $A\subset M$ metrického prostoru $(M,d)$ je kompaktní, právě když je topologicky kompaktní.
\end{theorem}
\begin{proof}
    Bez újmy na obecnosti můžeme vzít $A = M$.
    \begin{itemize}
        \item {
            \textbf{Implikace $\Rightarrow$:}\\
            Nechť $(M,d)$ je kompaktní metrický prostor a \[ M = \bigcup_{i\in I} X_i \]
            je jeho otevřené pokrytí, takže každá množina $X_i$ je otevřená.
            Nalezneme jeho konečné podpokrytí. Nejprve dokážeme, že
            \[ \forall \delta > 0 \, \exists \,\, \text{konečná množina} \,\, S_{\delta} \subset M: \bigcup_{a\in S_{\delta}} B(a,\delta) = M. \]
            Kdyby to tak nebylo, pak by existovalo $\delta_0 > 0$ a posloupnost $(a_n) \subset M$, že $m < n \Rightarrow d(a_m,a_n) \geq \delta_0$
            --- ve sporu s předpokládanou kompaktností množiny $M$ tato posloupnost nemá konvergentní podposloupnost.
            Skutečně, kdyby(negujeme hořejší tvrzení o $\delta$ a $S_\delta$) existovalo $\delta_0 > 0$, že pro každou konečnou
            množinu $S\subset M$ je
            \[ M \backslash \bigcup_{a\in S} B(a,\delta_0) \neq \emptyset, \]
            pak --- máme-li již definované body $a_1,a_2,\dots,a_n$ s $d(a_i, a_j) \geq \delta_0$ pro každé
            $1 \leq i < j \leq n$ --- vezmeme $a_{n+1} \in M\backslash \bigcup_{i=1}^{n} B(a_i, \delta_0)$
            a $a_{n+1}$ má od každého bodu $a_1,a_2,\dots, a_n$ vzdálenost alespoň $\delta_0$.
            Tak definujeme celou posloupnost $(a_n)$.

            Pro spor nyní předpokládejme, že hořejší otevřené pokrytí množiny $M$ množinami $X_i$ nemá konečné podpokrytí.
            Tvrdíme, že odtud vyplývá, že \[ \forall n \in\mathbb{N} \, \exists b_n \in S_{\frac{1}{n}} \, \forall i \in I: B\left(b_n, \frac{1}{n}\right) \not\subset X_i. \]
            Kdyby to tak nebylo, pak(negujeme předchozí tvrzení) by existovalo $n_0 \in\mathbb{N}$, že pro každé $b\in S_{\frac{1}{n_0}}$ existuje $i_b \in I$, že
            $B\left(b,\frac{1}{n_0} \right) \subset X_{i_b}$. Pak ale, protože $M = \bigcup_{b\in S_{\frac{1}{n_0}}} B\left( b, \frac{1}{n_0} \right)$, dávají
            indexy $J = \{ i_b \mid b\in S_{\frac{1}{n_0}} \} \subset I$ ve sporu s předpokladem konečné podpokrytí množiny $M$.

            Na samostatném řádku uvedené tvrzení o $n$ a $b_n$ tak platí a lze vzít posloupnost $(b_n)\subset M$. Podle předpokladu má konvergentní
            podposloupnost $b_{k_n}$ s $b := \lim b_{k_n} \in M$. Protože $X_i$ pokrývají $M$, existuje $j\in I$, že $b\in X_j$.
            Díky otevřenosti $X_i$ existuje $r > 0$, že $B(b,r)\subset X_j$. Vezmeme tak velké $n \in \mathbb{N}$, že $\frac{1}{k_n} < \frac{r}{2}$
            a $d(b,b_{k_n}) < \frac{r}{2}$. Pro každé $x\in B\left(b_{k_n}, \frac{1}{k_n}\right)$ pak podle $\Delta$-ové nerovnosti máme, že
            $d(x,b)\leq d(x,b_{k_n}) + d(b_{k_n}, b) < \frac{r}{2} + \frac{r}{2} = r.$
            Tedy \[ B\left( b_{k_n}, \frac{1}{k_n}\right)  \subset B(b,r) \subset X_j  , \]
            ve sporu s hořejší vlastností bodů $b_n$. Předpoklad, že konečné podpokrytí neexistuje, vede ke sporu.
            Proto pokrytí $M$ množinami $X_i, i\in I$, má konečné podpokrytí.
        }
        \item {
            \textbf{Implikace $\Leftarrow$:}\\
            Předpokládáme, že každé otevřené pokrytí množiny $M$ má konečné podpokrytí a odvodíme z toho, že
            každá posloupnost $(a_n) \subset M$ má konvergentní podposloupnost.
            Nejprve ukážeme, že předpoklad
            \[ \forall b \in M \, \exists r_b > 0: M_b := \{ n\in\mathbb{N}\mid a_n \in B(b,r_0) \} \,\,\,\text{je konečná} \]
            vede ke sporu. Z pokrytí $M = \bigcup_{b\in M} B(b,r_b)$ bychom totiž vybrali konečné podpokrytí dané
            konečnou množinou $N \subset M$ a nahlédli, že existuje $n_0$, že $n\geq n_0 \Rightarrow a_n \notin \bigcup_{b\in N} B(b,r_b)$,
            protože množina indexů $\bigcup_{b\in N}M_b$ je konečná (konečné sjednocení konečných množin). To je spor,
            protože $\bigcup_{b\in N}B(b,r_b) = M$. Předpoklad tedy neplatí a naopak je pravda, že
            \[ \exists b \in M \, r > 0: M_r := \{ n \in N \mid a_n \in B(b,r) \}\,\,\,\text{je nekonečná}. \]
            Teď už lehce z $(a_n)$ vybereme konvergentní podposloupnost $(a_{k_n})$ a limitou $b$. Nechť už jsme definovali
            indexy $1 \leq k_1 < k_2 < \dots < k_n$, že $d(b,a_{k_i}) < \frac{1}{i}$ pro $i = 1,2,\dots,n$. Množina indexů
            $M_{\frac{1}{n+1}}$ je nekonečná, takže můžeme zvolit takové $k_{n+1} \in \mathbb{N}$, že $k_{n+1} > k_n$
            a $k_{n+1} \in M_{\frac{1}{n+1}}$. Pak i $d(b,a_{k_{n+1}}) < \frac{1}{n+1}$. Takto je definována posloupnost
            $(a_{k_n})$ konvergující k $b$.
        }
    \end{itemize}
\end{proof}

\section{Dokažte existenci $n$-tých odmocnin v $\mathbb{C}$.}

\begin{remark}
    Komplexní jednotková kružnice \[ S := \{ z \in \mathbb{C} \mid |z| = 1 \} \subset \mathbb{C} \]
    je souvislá množina.
\end{remark}

\begin{lemma}
    Pro každé nezáporné $x \in \mathbb{R}$ a každé $n \in \mathbb{N}$ existuje nezáporné $y\in \mathbb{R}$ takové, že $y^n = x$.
\end{lemma}

\begin{lemma}[Druhá odmocnina v $\mathbb{C}$]
    $\forall a+bi \in \mathbb{C}$ máme pro vhodnou volbu znamének v reálných číslech
    \[ c := \pm \frac{\sqrt{\sqrt{a^2 + b^2} + a}}{\sqrt{2}} \hspace{5mm} a \hspace{5mm} d := \pm \frac{\sqrt{\sqrt{a^2 + b^2} - a}}{\sqrt{2}}, \]
    že $(c+di)^2 = a+bi$.
\end{lemma}

Z předchozích dvou tvrzení lze dokázat, že pokud pro každé $u \in S$ a pro každé liché $ n \in\mathbb{N}$ $\exists v \in S: v^n = u$, pak
platí následující věta.

\begin{theorem}[$n$-té odmocniny v $\mathbb{C}$]
    Komplexní čísla obsahují všechny $n$-té odmocniny, tedy
    \[ \forall u \in \mathbb{C} \hspace{1mm} \forall n \in \mathbb{N} \hspace{1mm} \exists v \in \mathbb{C}: v^n = u. \]
\end{theorem}
\begin{proof}
    \LARGE
    \textbf{TODO}
\end{proof}

\section{Dokažte Besselovu nerovnost.}

\begin{lemma}[Ortogonalita sinů a cosinů]
    Pro každá dvě celá čísla $m,n\geq 0$ je \[ \left< \sin(mx),\cos(nx) \right> = 0. \]
    Pro každá dvě delá čísla $m,n\geq 0$, kromě $m=n=0$, je
    \[ \left< \sin(mx), \sin(nx) \right> = \left< \cos(mx), \cos(nx) \right> =
    \begin{cases} \pi \hspace{2mm} \dots \hspace{2mm} m=n\\ 0 \hspace{2mm} \dots \hspace{2mm} m\neq n. \end{cases} \]
    Konečně
    \[ \left< \sin(0x), \sin(0x) \right> = 0 \hspace{5mm} a \hspace{5mm} \left< \cos(0x), \cos(0x) \right> = 2\pi. \]
\end{lemma}

\begin{definition}[Kosinové a sinové Fourierovy koeficienty]
    Pro každou funkci $f \in \mathcal{R}(-\pi,\pi)$ definujeme její
    \underline{kosinové} Fourierovy koeficienty
    \[ a_n := \frac{\left< f(x), \cos(nx) \right>}{\pi} = \frac{1}{\pi}\int_{-\pi}^{\pi}f(x)\cos(nx)\,\text{dx}, n = 0,1,\dots \]
    a \underline{sinové} Fourierovy koeficienty
    \[ b_n := \frac{\left< f(x), \sin(nx) \right>}{\pi} = \frac{1}{\pi}\int_{-\pi}^{\pi}f(x)\sin(nx)\,\text{dx}, n = 1,2,\dots \]
\end{definition}

\begin{definition}[Fourierova řada funkce]
    Fourierova řada funkce $f$($\in \mathcal{R}(-\pi,\pi)$) je trigonometrická řada
    \[ F_f(x) := \frac{a_0}{2} + \sum_{n=1}^{\infty} \left( a_n \cos(nx) + b_n \sin(nx) \right), \]
    kde $a_n$ a $b_n$ jsou po řadě její kosinové a sinové Fourierovy koeficienty.
\end{definition}

\noindent
Geometricky nahlíženo, pracujeme v nekonečně rozměrném vektorovém prostoru se (skoro) skalárním součinem $\left<\cdot,\cdot\right>$,
v němž jsou \uv{souřadnými osami}(prvky ortogonální báze) funkce
\[ \{ \cos(nx) \mid n\in \mathbb{N}_0 \} \cup \{ \cos(nx) \mid n\in \mathbb{N} \} \]
V kontrastu s kartézskými souřadnicemi bodů v $\mathbb{R}^n$ se ale zdaleka ne každá funkce rovná součtu
své Fourierovy řady.

\begin{theorem}[Besselova nerovnost]
    Pro Fourierovy koeficienty $a_n$ a $b_n$ funkce $f \in \mathcal{R}(-\pi,\pi)$ platí nerovnost
    \[ \frac{a_0^2}{2} + \sum_{n=1}^{\infty}(a_n^2+b_n^2) \leq \frac{\left<f,f\right>}{\pi} = \frac{1}{\pi}\int_{-\pi}^{\pi}f^2. \]
\end{theorem}
\begin{proof}
    \LARGE
    \textbf{TODO}
\end{proof}

\section{Spočítejte, že $ \sum_{n=1}^{\infty} \frac{1}{n^2} = \frac{\pi^2}{6} $}

\begin{example}[Basilejský problém]
    \LARGE
    TODO
\end{example}

\section{Dokažte, že stejnoměrná limita spojitých funkcí je spojitá funkce.}



\section{Dokažte případ 2 nebo případ 3 Pólyovy věty.}
\section{Dokažte, že $\rho \neq 0$.}
\section{Dokažte Caychy-Goursatovu větu pro obdélníky.}
\section{Vyřešte diferenciální rovnici $y' + ay = b$.}


\end{document}
