\documentclass[11pt]{article}

\usepackage[czech]{babel}
\usepackage{a4wide}
\usepackage[utf8]{inputenc}
\usepackage[T1]{fontenc}
\usepackage{fancyhdr}
\usepackage{amssymb}
\usepackage{amsthm}
\usepackage{amsmath}
\usepackage{mathtools}
\usepackage{mleftright}
\usepackage{subfig}

% forcing footnotes to be at the very bottom
\usepackage[bottom]{footmisc}

\usepackage{hyperref}
\usepackage{titlesec}
%This has to be the last
\usepackage{subfiles}

\usepackage{geometry}
\geometry{
    a4paper,
    total={170mm,257mm},
    right=20mm,
    left=20mm,
    top=30mm,
    bottom=20mm,
}

\DeclareMathOperator{\rank}{rank}
\DeclareMathOperator{\Span}{span}

\newtheoremstyle{nontheoremstyle}{1em}{1em}{}{}{\bfseries}{:}{.5em}{}
\newtheoremstyle{theoremstyle}{1em}{1em}{\it}{}{\bfseries}{:}{.5em}{}

\theoremstyle{nontheoremstyle}
\newtheorem*{definition}{Definice}
\newtheorem*{example}{Příklad}
\renewenvironment{proof}{{\noindent\bfseries Důkaz:}}{\qed}
\newtheorem*{intuition}{Intuice}
\newtheorem*{remark}{Poznámka}
\newtheorem*{consequence}{Důsledek}
\newtheorem*{observation}{Pozorování}

% ošklivý hack k tomu, aby environment 'definitionnodot' neměl na konci . nebo :
% hodí se, když chceme Dělat něco jako „Definice (Riemannův integrál) je funkce...“
\newtheoremstyle{nontheoremstylenodot}{1em}{1em}{}{}{\bfseries}{}{.3em}{}
\theoremstyle{nontheoremstylenodot}
\newtheorem*{definitionnodot}{Definice}

\theoremstyle{theoremstyle}
\newtheorem*{theorem}{Věta}
\newtheorem*{lemma}{Tvrzení}

\titleformat{\section} {\normalfont\fontsize{16}{15}\bfseries}{\thesection}{1em}{}
\titleformat{\subsection} {\normalfont\fontsize{14}{15}\bfseries}{\thesubsection}{1em}{}
\titleformat{\subsubsection} {\normalfont\fontsize{12}{15}\bfseries}{\thesubsubsection}{1em}{}

\pagestyle{fancy}
\fancyhf{}
\rhead{Matematická analýza III}
\lhead{Otázky na Letní semestr 2020/2021}
\fancyfoot{}
\fancyfoot[R]{\thepage}

\begin{document}

\section{Definujte metrický prostor a sférickou metriku. Dokažte, že hemisféra není plochá.}

\begin{definition}[Metrický prostor]
    Metrický prostor je dvojice $(M,d)$ množiny $M\neq \emptyset$ a zobrazení \[d: M \times M \to \mathbb{R}\]
    zvaného \underline{metrika} či \underline{vzdálenost}, které $\forall x,y,z \in M$ splňuje:
    \begin{enumerate}
        \item $d(x,y) = 0 \iff x=y$
        \item $d(x,y) = d(y,x)$
        \item $d(x,y) \leq d(x,z) + d(z,y)$
    \end{enumerate}
    Z těchto podmínek plyne i $d(x,y) \geq 0$.
\end{definition}

\begin{example}[Sférická metrika]
    Jako \[S := \{ (x_1,x_2,x_3) \in \mathbb{R}^3 \mid x_1^2 + x_2^2 + x_3^3 = 1 \}\] si označíme \underline{jednotkovou sféru} v
    euklidovském prostoru $\mathbb{R}^n$. Funkci $s: S\times S \to [0,\pi]$ definujeme pro $\overline{x}, \overline{y} \in S$
    jako \[ s(\overline{x}, \overline{y}) = \begin{cases}
        0 \dots \overline{x} = \overline{y}\\
        \varphi \dots \overline{x} \neq \overline{y}
    \end{cases} \]
    kde $\varphi$ je úhel sevřený dvěma polopřimkami procházejícímí počátkem $\overline{0}$ a body $\overline{x}$ a $\overline{y}$.
    Tento úhel je vlastně délka kratšího z oblouků mezi body $\overline{x}$ a $\overline{y}$ na jednotkové kružnici vytknuté na $S$ rovinou určenou
    počátkem a body $\overline{x}$ a $\overline{y}$. Funkci $s$ nazveme sférickou metrikou.
\end{example}

\begin{theorem}[$H$ není plochá]
    Metrický prostor $(H,s)$ není izometrický žádnému Euklidovskému prostoru $(X,e_n)$ s $X \subset \mathbb{R}^n$    
\end{theorem}
\begin{proof}
    \LARGE
    \textbf{TODO}
\end{proof}

\section{Dokažte Ostrowskiho větu.}

\begin{definition}[$p$-adický řád]
    Nechť $p \in \{ 2,3,5,7,11,\dots \}$ je prvočíslo a nechť $n \in \mathbb{Z}$ je nenulové celé číslo.
    Jako $p$-adický řád čísla $n$ definujeme
    \[ \text{ord}_p(n) := \max(\{ m \in \mathbb{N}_0 : p^m \mid n \})\footnote{$\cdot \mid \cdot$ značí relaci dělitelnosti.} \]
    Dále ještě $\forall p$ definujeme $\text{ord}_p(0) := +\infty$.
\end{definition}

\begin{remark}[Rozšíření $\text{ord}_p(\cdot)$ na zlomky]
    Pro nenulové $\alpha = \frac{a}{b} \in \mathbb{Q}$ definujeme
    \[ \text{ord}_p(\alpha) := \text{ord}_p(a) - \text{ord}_p(b) \]
    Jinak opět $\text{ord}_p(0) = \text{ord}_p(\frac{0}{b}) := +\infty$.
\end{remark}

\begin{definition}[$p$-adická norma]
    Fixujeme reálnou konstantu $c \in (0,1)$ a definujeme funkci $\lvert \cdot \rvert_p : \mathbb{Q} \to [0, +\infty)$ jako
    \[ \left| \frac{a}{b} \right|_p := c^{\text{ord}_p\left( \frac{a}{b} \right)} \]
    kde klademe $\lvert 0 \rvert_p = c^{+\infty} := 0$
\end{definition}

\begin{definition}[Kanonická $p$-adická norma]
    Pro $\alpha \in\mathbb{Q}$ a prvočíslo $p$ je kanonická $p$-adická norma $|| \cdot ||_p$ definovaná jako
    \[|| \alpha ||_p := p^{-\text{ord}_p(\alpha)}\]
    to jest v obecné $p$-adické normě $|| \cdot ||_p$ klademe $c := \frac{1}{p}$.
\end{definition}

\begin{theorem}[A. Ostrowski]
    Nechť $||\cdot||$ je norma na tělese racionálních čísel $\mathbb{Q}$. Pak nastává jedna ze tří následujících možností.
    \begin{enumerate}
        \item Je to triviální norma.
        \item Existuje reálné $c\in (0,1]$ takové, že $||x|| = |x|^c$.
        \item Existuje reálné $c \in (0,1)$ a prvočíslo $p$, že $||x|| = |x|_p = c^{\text{ord}_p(x)}$(kde $c^\infty := 0$).
    \end{enumerate}
    Modifikovaná absolutní hodnota a $p$-adické normy jsou tedy jediné netriviální
    normy na tělese racionálních čísel.
\end{theorem}
\begin{proof}
    \LARGE
    \textbf{TODO}
\end{proof}

\section{Dokažte Heine-Borelovu větu.}

\begin{lemma}[Topologická spojitost]
    Nechť $f: M \to N$ je zobrazení mezi metrickými prostory $(M,d)$ a $(N,e)$. prostorem
    \[ f \hspace{1mm} \text{je spojité} \iff \forall \hspace{1mm} \text{OM} \hspace{1mm} A \subset N: f^{-1}[A] = \{ x\in M \mid f(x) \in A \} \subset M \hspace{1mm} \text{je OM}.
    \footnote{OM zkracuje sousloví \uv{otevřená množina}. } \]
\end{lemma}

\noindent
Toto tvrzení platí i pro uzavřené množiny.

\begin{lemma}[Spojitý obraz kompaktu]
    Nechť $(M,d)$ a $(N,e)$ jsou metrické prostory, $X\subset M$ je neprázdná kompaktní množina a \[ f: X\to N \] je spojitá funkce.
    Pak obraz $f[X] \subset N$ je kompaktní množina.
\end{lemma}

\begin{lemma}[Spojitost inverzu]
    Nechť $f: X\to N$ je spojité zobrazení z neprázdné kompaktní množiny $X \subset M$ v metrickém prostoru $(M,d)$ do $(N,e)$.
    Potom inverzní zobrazení \[ f^{-1}: f[X] \to X \] je spojité.
\end{lemma}

\begin{definition}[Homeomorfismus]
    Zobrazení $f: M\to N$ mezi metrickými prostory $(M,d)$ a $(N,e)$ je jejich homeomorfismus, je-li $f$ bijekce a jsou-li $f$ a $f^{-1}$ spojitá zobrazení.
    Pokud mezi $(M,d)$ a $(N,e)$ existuje homeomorfismus, jsou \underline{homeomorfní}.
\end{definition}

\begin{definition}[Topologická kompaktnost]
    Podmnožina $A\subset M$ metrického prostoru $(M,d)$ je topologicky kompaktní, pokud každý systém otevřených množin $\{ X_i \mid i \in I \}$ v $M$ platí:
    \[ \bigcup_{i \in I} X_i \supset A \Rightarrow \exists \hspace{1mm} \text{konečná množina} \hspace{1mm} J \subset I: \bigcup_{i\in J}X_i \supset A. \]
\end{definition}

\begin{theorem}[Heine-Borelova]
    Podmnožina $A\subset M$ metrického prostoru $(M,d)$ je kompaktní, právě když je topologicky kompaktní.
\end{theorem}
\begin{proof}
    \LARGE
    \textbf{TODO}
\end{proof}

\section{Dokažte existenci $n$-tých odmocnin v $\mathbb{C}$.}

\begin{remark}
    Komplexní jednotková kružnice \[ S := \{ z \in \mathbb{C} \mid |z| = 1 \} \subset \mathbb{C} \]
    je souvislá množina.
\end{remark}

\begin{lemma}
    Pro každé nezáporné $x \in \mathbb{R}$ a každé $n \in \mathbb{N}$ existuje nezáporné $y\in \mathbb{R}$ takové, že $y^n = x$.
\end{lemma}

\begin{lemma}[Druhá odmocnina v $\mathbb{C}$]
    $\forall a+bi \in \mathbb{C}$ máme pro vhodnou volbu znamének v reálných číslech
    \[ c := \pm \frac{\sqrt{\sqrt{a^2 + b^2} + a}}{\sqrt{2}} \hspace{5mm} a \hspace{5mm} d := \pm \frac{\sqrt{\sqrt{a^2 + b^2} - a}}{\sqrt{2}}, \]
    že $(c+di)^2 = a+bi$.
\end{lemma}

Z předchozích dvou tvrzení lze dokázat, že pokud pro každé $u \in S$ a pro každé liché $ n \in\mathbb{N}$ $\exists v \in S: v^n = u$, pak
platí následující věta.

\begin{theorem}[$n$-té odmocniny v $\mathbb{C}$]
    Komplexní čísla obsahují všechny $n$-té odmocniny, tedy
    \[ \forall u \in \mathbb{C} \hspace{1mm} \forall n \in \mathbb{N} \hspace{1mm} \exists v \in \mathbb{C}: v^n = u. \]
\end{theorem}
\begin{proof}
    \LARGE
    \textbf{TODO}
\end{proof}

\section{Dokažte Besselovu nerovnost.}

\begin{lemma}[Ortogonalita sinů a cosinů]
    Pro každá dvě celá čísla $m,n\geq 0$ je \[ \left< \sin(mx),\cos(nx) \right> = 0. \]
    Pro každá dvě delá čísla $m,n\geq 0$, kromě $m=n=0$, je
    \[ \left< \sin(mx), \sin(nx) \right> = \left< \cos(mx), \cos(nx) \right> =
    \begin{cases} \pi \hspace{2mm} \dots \hspace{2mm} m=n\\ 0 \hspace{2mm} \dots \hspace{2mm} m\neq n. \end{cases} \]
    Konečně
    \[ \left< \sin(0x), \sin(0x) \right> = 0 \hspace{5mm} a \hspace{5mm} \left< \cos(0x), \cos(0x) \right> = 2\pi. \]
\end{lemma}

\begin{definition}[Kosinové a sinové Fourierovy koeficienty]
    Pro každou funkci $f \in \mathcal{R}(-\pi,\pi)$ definujeme její
    \underline{kosinové} Fourierovy koeficienty
    \[ a_n := \frac{\left< f(x), \cos(nx) \right>}{\pi} = \frac{1}{\pi}\int_{-\pi}^{\pi}f(x)\cos(nx)\,\text{dx}, n = 0,1,\dots \]
    a \underline{sinové} Fourierovy koeficienty
    \[ b_n := \frac{\left< f(x), \sin(nx) \right>}{\pi} = \frac{1}{\pi}\int_{-\pi}^{\pi}f(x)\sin(nx)\,\text{dx}, n = 1,2,\dots \]
\end{definition}

\begin{definition}[Fourierova řada funkce]
    Fourierova řada funkce $f$($\in \mathcal{R}(-\pi,\pi)$) je trigonometrická řada
    \[ F_f(x) := \frac{a_0}{2} + \sum_{n=1}^{\infty} \left( a_n \cos(nx) + b_n \sin(nx) \right), \]
    kde $a_n$ a $b_n$ jsou po řadě její kosinové a sinové Fourierovy koeficienty.
\end{definition}

\noindent
Geometricky nahlíženo, pracujeme v nekonečně rozměrném vektorovém prostoru se (skoro) skalárním součinem $\left<\cdot,\cdot\right>$,
v němž jsou \uv{souřadnými osami}(prvky ortogonální báze) funkce
\[ \{ \cos(nx) \mid n\in \mathbb{N}_0 \} \cup \{ \cos(nx) \mid n\in \mathbb{N} \} \]
V kontrastu s kartézskými souřadnicemi bodů v $\mathbb{R}^n$ se ale zdaleka ne každá funkce rovná součtu
své Fourierovy řady.

\begin{theorem}[Besselova nerovnost]
    Pro Fourierovy koeficienty $a_n$ a $b_n$ funkce $f \in \mathcal{R}(-\pi,\pi)$ platí nerovnost
    \[ \frac{a_0^2}{2} + \sum_{n=1}^{\infty}(a_n^2+b_n^2) \leq \frac{\left<f,f\right>}{\pi} = \frac{1}{\pi}\int_{-\pi}^{\pi}f^2. \]
\end{theorem}
\begin{proof}
    \LARGE
    \textbf{TODO}
\end{proof}

\section{Spočítejte, že $ \sum_{n=1}^{\infty} \frac{1}{n^2} = \frac{\pi^2}{6} $}

\begin{example}[Basilejský problém]
    \LARGE
    TODO
\end{example}

\section{Dokažte, že stejnoměrná limita spojitých funkcí je spojitá funkce.}



\section{Dokažte případ 2 nebo případ 3 Pólyovy věty.}
\section{Dokažte, že $\rho \neq 0$.}
\section{Dokažte Caychy-Goursatovu větu pro obdélníky.}
\section{Vyřešte diferenciální rovnici $y' + ay = b$.}


\end{document}
